\section{Diskussion}
Bei der Messung zum Verstärkungsfaktor fällt im Fall $R_N = \SI{1}{\kilo \ohm}$ (Abbildung \ref{fig:lin1}) auf, dass die letzten fünf Werte um einen festen Faktor größer als die erwartete Kurve sind.
Abgesehen davon hat die Kurve den gleichen Verlauf wie der aus der  Literatur bekannte Frequenzgang des Linearverstärkers.

Der Frequenzgang lässt sich in drei Bereiche aufteilen.
Bis etwa zur Grenzfrequenz ist der Verstärkungsfaktor konstant.
Daraufhin fällt er exponenziell ab und wird letzendlich wieder konstant.
Die konstante Verstärkungsfaktor deckt sich mit den Erwartungen für einen idealen Operationsverstärker.
Der Ausdruck $\frac{R_N}{R_1}$ wird frequenzabhängig, wenn man annimmt, dass der Widerstand $R_N$ eine parasitäre Kapazität hat.
In diesem Fall wäre $V' \propto \frac{1}{\nu}$.
In einem Ersatzschaltbild würde ein Kondensator parallel zum Widerstand $R_N$ geschaltet.

Die bestimmten Werte für das Verstärkung-Bandbreite-Produkt in Tabelle \ref{tab:vbp} sind mit einer gewissen Abweichung gleich groß.
Dies bestätigt den erwarteten Zusammenhang, dass das Verstärkung-Bandbreite-Produkt konstant ist.

Weiterhin wurde der reale Operationsverstärker durch die Berücksichtigung einer Leerlaufverstärkung angenähert.
Es fällt auf, dass die in Tabelle \ref{tab:leerlauf} dargestellten Werte für $V$ proportional zu der Verstärkung der Schaltung sind.
Das ist ein unerwartetes Ergebnis, da die Leerlaufverstärkung eine Kenngröße des OPs ist und unabhängig von der Beschaltung sein sollte.
Die Abweichung vom theoretischen Wert, die proportional zur Verstärkung ist, könnte besser durch den endlichen Ausgangswiderstand des OPs erklärt werden.

Bei dem Integrierer wurde die Beziehung $U_A \propto \frac{1}{\omega}$ überprüft.
Eine Fitfunktion mit sehr kleinem Fehler konnte bestimmt werden.
Die mit dem Oszilloskop untersuchte integrierte Rechteckspannung ist wie erwartet ein Dreieckssignal.
Das Signal ist unten abgeschnitten, was dadurch zu erklären ist, dass der OP in Sättigung geht.
Die Amplitude der Ausgangsspannung ist in der Größenordnung der Betriebsspannung des OPs.
Die anderen untersuchten Spannungen entsprechen nur grob den erwarteten Signalen.
Es fällt auf, dass der Integrierer für die Dreieck- und Sinusspannung das gleiche Signal ausgibt.

Die Sinusspannung wird vom Differenzierer korrekterweise in eine phasenverschobene Sinusspannung umgewandelt.
Allerdings wird die Rechteckspannung nicht korrekt differenziert.
Die Dreieckspannung hingegen führt zu einem stark rauschenden, aber korrekten Signal.

Bei den Thermodrucken zum Logarithmierer und Exponenzierer fällt auf, dass Ausgangssignale invertiert sind.
Beide Signale fallen mit steigender Eingangsspannung, obwohl Die Logarithmus- und Exponentialfunktion steigen.
Abgesehen davon ist beim Exponenzierer eine e-Funktion erkennbar.
Der Logarithmierer geht für einen großen Teil der Kurve in Sättigung, was auf eine ungeschickte Wahl der Betriebsspannung und Eingangsamplitude schließen lässt.