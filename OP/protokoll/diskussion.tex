\section{Diskussion}
Bei der Messung zum Verstärkungsfaktor fällt im Fall $R_N = \SI{1}{\kilo \ohm}$ (Abbildung \ref{fig:lin1}) auf, dass die letzten fünf Werte um einen festen Faktor größer als die erwartete Kurve sind.
Abgesehen davon hat die Kurve den gleichen Verlauf wie der aus der  Literatur bekannte Frequenzgang des Linearverstärkers.

Der Frequenzgang lässt sich in drei Bereiche aufteilen.
Bis etwa zur Grenzfrequenz ist der Verstärkungsfaktor konstant.
Daraufhin fällt er exponenziell ab und wird letzendlich wieder konstant.
Die konstante Verstärkungsfaktor deckt sich mit den Erwartungen für einen idealen Operationsverstärker.
Der Ausdruck $\frac{R_N}{R_1}$ wird frequenzabhängig, wenn man annimmt, dass der Widerstand $R_N$ eine parasitäre Kapazität hat.
In diesem Fall wäre $V' \propto \frac{1}{\nu}$.
In einem Ersatzschaltbild würde ein Kondensator parallel zum Widerstand $R_N$ geschaltet.

Die bestimmten Werte für das Verstärkung-Bandbreite-Produkt in Tabelle \ref{tab:vbp} sind mit einer gewissen Abweichung gleich groß.
Dies bestätigt den erwarteten Zusammenhang, dass das Verstärkung-Bandbreite-Produkt konstant ist.

Weiterhin wurde der reale Operationsverstärker durch die Berücksichtigung einer Leerlaufverstärkung angenähert.
Es fällt auf, dass die in Tabelle \ref{tab:leerlauf} dargestellten Werte für $V$ proportional zu der Verstärkung der Schaltung sind.
Das ist ein unerwartetes Ergebnis, da die Leerlaufverstärkung eine Kenngröße des OPs ist und unabhängig von der Beschaltung sein sollte.
Die Abweichung vom theoretischen Wert, die proportional zur Verstärkung, könnte besser durch den endlichen Ausgangswiderstand des OPs erklärt werden.