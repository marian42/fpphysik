\section{Theorie}

Der Operationsverstärker ist eine integrierte Schaltung, die aus mehreren Transistoren zusammen gesetzt ist.
Der Name ergibt sich daraus, dass mit einem Operationsverstärker durch geschickte Beschaltung eine Vielzahl mathematischer Operationen realisiert werden kann.

Die Ausgangsspannung ergibt sich aus der Differenz der Eingangsspannungen, die verstärkt und auf die Betriebsspannungen begrenzt ist.

\subsection{Der ideale Operationsverstärker}

\begin{figure}
	\centering
	\includegraphics[width=0.6\textwidth]{img/op.png}
	\caption{Schaltsymbol des Operationsverstärkers \cite{v51}}
	\label{fig:op}
\end{figure}

Das Schaltbild eines Operationsverstärkers is in Abbildung \ref{fig:op} dargestellt.
Der OP hat zwei Eingänge, an denen die Spannungen $U_p$ und $U_N$ anliegen.
Die Spannung am Ausgang heißt $U_A$.
An zwei weiteren Eingängen wird die positive und negative Betriebsspannung $U_B$ angelegt.

Die Ausgangsspannung ist die verstärkte Differenz der Eingangsspannungen, also
\begin{align}
	U_A = V \left(U_p - U_N\right).
\end{align}
Dabei ist $V$ die Leerlaufverstärkung.
Eine Erhöhung von $U_n$ führt zu einer Verringerung der Ausgangsspannung.
Deswegen wird der Eingang von $U_n$ auch invertierender Eingang genannt.
Der andere Eingang heißt entsprechend nicht-invertierender Eingang.

Die Ausgangsspannung ist auf die Höhe der Betriebsspannung begrenzt, das heißt:
\begin{align}
	-U_B < U_A < U_B
\end{align}

Daraus ergibt sich die Kennlinie des Operationsverstärkers, die in Abbildung \ref{fig:kennlinie} dargestellt ist.
\begin{figure}
	\centering
	\includegraphics[width=0.5\textwidth]{img/kennlinie.png}
	\caption{Kennlinie des Operationsverstärkers \cite{v51}}
	\label{fig:kennlinie}
\end{figure}

Beim idealen Bauteil ist $V \rightarrow \infty$.
Folglich nimmt die Ausgangsspannung immer $\pm U_B$ an und der OP arbeitet als Komparator.
Weiterhin fließt beim idealen Bauteil kein Strom in die Eingänge.
Dies entspricht einem unendlich großen Eingangswiderstand $r_e$.
Der Ausgang hingegen kann einen beliebig großen Strom liefern, entsprechend eines Ausgangswiderstands $r_a$ von $0$.

\subsection{Der reale Operationsverstärker}

Beim realen Operationsverstärker nehmen die Leerlaufverstärkung, die Eingangswiderstände und der Ausgangswiderstand endliche Werte an.
Es gibt einige weitere Effekte, die im Folgenden beschrieben werden.

\subsubsection{Gleichtaktverstärkung}
Die Gleichtaktverstärkung ist die Spannung, die am Ausgang gemessen wird, wenn an beiden Eingängen die gleiche Spannung $U_{Gl}$ angelegt wird.
Sie ist idealerweise $0$.
\begin{align}
	V_{Gl} = \frac{\Delta U_A}{\Delta U_{Gl}}
\end{align}

\subsubsection{Eingangswiderstand}

Da die Eingangswiderstände nicht unendlich sind, fließen in die Eingänge die Ströme $I_p$ und $I_N$.
Der Differenzeingangswiderstand wird definiert als
\begin{align}
	r_D =
	\begin{cases}
		\frac{\Delta U_p}{\Delta I_p} \quad & \text{für} \quad U_n = 0\\
		\frac{\Delta U_N}{\Delta I_N} \quad & \text{für} \quad U_p = 0
	\end{cases}
\end{align}
und der Gleichtakteingangswiderstand als 
\begin{align}
	r_{Gl} = \frac{\Delta U_{Gl}}{\Delta I_{Gl}},
\end{align}
mit $Up = U_N = U_{Gl}$ und $I_{Gl} = I_p + I_N$.

\subsubsection{Offsetspannung}
Legt man am realen OP an beiden Eingängen die Spannung $0$ an, ist die Ausgangsspannung nicht wie erwartet $0$.
Die Offsetspannung ist definiert als die Differenz der Eingangsspannungen wenn die Ausgangsspannung $0$ ist.
\begin{align}
	U_0 = U_p - U_N
\end{align}
Diese Größe ist von äußeren Faktoren wie der Temperatur, der Zeit und den Betriebsspannungen abhängig.

\subsection{Schaltungen mit Operationsverstärkern}

\subsubsection{Linearverstärker}

\subsubsection{Umkehr-Integrierer}

\subsubsection{Umkehr-Differenzierer}

\subsubsection{Schmitt-Trigger}

\subsubsection{Oszillator}

\subsubsection{Sinusgenerator}

\subsubsection{Logarithmierer und Exponenzierer}

