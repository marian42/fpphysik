\section{Theorie}
\subsection{Bindungstypen in Kristallen}
Da in der STM ausschließlich Festkörper untersucht werden, ist es von besonderer Wichtigkeit, ein Verständnis über die Bindungstypen und Kräfte in Kristallen zu entwickeln. Dieser Abschnitt liefert einen Abriss über die wichtigsten Kräfte und Bindungen in Festkörpern.

\subsubsection{}
\subsubsection{}
\subsubsection{}
\subsubsection{}
\subsubsection{}

\subsection{Tunneleffekt}
Die Funktionsweise des Rastertunnelmikroskops basiert auf dem Quantenmechanischen Phänomen des Tunneleffekts. Ist ein Wellenpaket von einer endlich hohen Potentialbarriere eingeschlossen, kann es diese im klassischen Fall nicht durchdringen. Quantenmechanisch ist dies jedoch möglich, da das Wellepaket durch die Lösung der Schrödingergleichung beschrieben werden kann, die im Betragsquadrat lediglich Information über die Aufenthaltswahrscheinlichkeit des Wellenpakets liefert. Diese Wahrscheinlichkeit ist außerhalb der Potentialbarriere ungleich null, sodass das Wellenpaket durchaus hinter der Barriere auffindbar sein kann. 

\noindent Bei der STM wird ein elektrisch leitender oder mit einem solchen Material überzogener Festkörper untersucht, indem eine Pt-Ir-Drahtspitze an eine Probe genähert wird, die dann über diese rastert. Es existiert also ein Valenzband in dem sich Elektronen der Fermienergie befinden. Die zu überwindende Potentialbarriere für den Tunneleffekt in diesem Versuch ist das Vakuum. Der Effekt ist nur für Abstände in atomischer Skala relevant, weswegen eine typische Probe der Größenordnung 100\,\r{A}\(^2\) ist und der Versuchsaufbau dementsprechend sensitiv gegenüber dem Abstand zwischen Spitze und Probe ist. Eine Genauigkeit von 1\,\r{A} für die Spitze kann beim Anlegen einer Spannung von \(0,1\)V erreicht werden.

\noindent Wird eine Spitze in die Nähe einer Probe mit diesen Eigenschaften gebracht, kann der Tunneleffekt für die Elektronen der Spitze oder für die der Probe eintreten, was davon abhänging ist, welches Vorzeichen die an die Spitze angelegte Spannung hat. Es entsteht ein Tunnelstrom, der gemessen werden kann.

\noindent Der Tunnelstrom fällt exponentiell mit dem Abstand \(d\) und der Wurzel der Austrittsarbeit \(\Phi\) des Festkörpers. Dies ist die Energie, die mindestens aufgewandt werden muss, um ein Elektron aus einem ungeladenen Festkörper zu lösen. 

\begin{equation}
I_\text{T}=\frac{U}{d}\exp{(-Kd\sqrt{\Phi})},\quad K\approx1,025\frac{1}{\text{\r{A}}\,\text{eV}^\frac12}
\end{equation}

\noindent Daraus ergeben sich direkt die zwei Hauptschwierigkeiten dieses Versuchs. Zum einen muss der gesamte Versuchsaufbau bezüglich mechanischer Vibrationen in der Größenordnung von \(<1\)\,\r{A} gedämpft werden, da der Abstand \(d\approx0,5\)\,\r{A} anfällig auf sehr schwache Schwingungen ist.  

\noindent Zum anderen muss die Drahtspitze möglichst spitz und sauber präpariert sein. Das heißt der Draht wird vor der benutzung mit 2-Propanol von lipopholem Dreck befreit.

\subsection{Fehlerquellen}
\subsubsection{Intrinsische Nichtlinearität}

\subsubsection{Hysterese}
\subsubsection{Kriechen}
Die Reatkion des Piezokristalls auf eine Spannungsänderung findet im Idealfall instantan statt.
In der Realität lässt sich die Bewegung in zwei Teile aufteilen.
Der erste Teil der Bewegung findet in weniger als einer Millisekunde statt, während der zweite Teil länger dauert.
Der Kriechfaktor gibt das Verhältnis der Strecken vom ersten und zweiten Teil an.

Eine Folge dieses Effekts ist, dass Messungen, die mit unterschiedlicher Geschwindigkeit durchgeführt werden, unterschiedliche Ergebnisse liefern.

\subsubsection{Alterung}
Die Ausrichtung der Dipole im Piezokristall wird durch Anlegen von Spannung beeinflusst.
Ebenso hebt sich die Ausrichtung der Dipole bei längerer Lagerung auf.
Als Folge davon ändert sich der Belastungskoeffizient des Piezokristalls und die Position auf der X, Y und Z-Achse wird verfälscht.

\subsubsection{Kreuzkopplung}

Die Aufhängung der Spitze besteht aus drei Piezo-Röhren, die am Ende verbunden sind.
Das Strecken oder Stauchen einer Röhre führt deshalb nicht, wie gewünscht, zu einer Translationsbewegung auf einer Achse, sondern zu einer Art Schwenkbewegung auf mehreren Achsen.
Die ungewünschte Bewegung entlang der anderen Achsen wird als Kreuzkopplung bezeichnet.

Die Kreuzkopplung wird entweder durch Software korrigiert oder die Spitze wird durch eine Regelschleife an die korrekte Stelle gefahren.

\subsection{Piezo-elektrizität}