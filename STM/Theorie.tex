\documentclass[paper=a4]{scrartcl}

\usepackage[german]{babel}
\usepackage{bookmark}
\usepackage{booktabs, dcolumn}
\usepackage{icomma}
\usepackage{amsmath, amssymb}
\usepackage[utf8]{inputenc}

\begin{document}
\section{Theorie}
\subsection{Bindungstypen in Kristallen}
\subsection{Tunneleffekt}
Die Funktionsweise des Rastertunnelmikroskops basiert auf dem Quantenmechanischen Phänomen des Tunneleffekts. Ist ein Wellenpaket von einer endlich hohen Potentialbarriere eingeschlossen, kann es diese im klassischen Fall nicht durchdringen. Quantenmechanisch ist dies jedoch möglich, da das Wellepaket durch die Lösung der Schrödingergleichung beschrieben werden kann, die im Betragsquadrat lediglich Information über die Aufenthaltswahrscheinlichkeit des Wellenpakets liefert. Diese Wahrscheinlichkeit ist außerhalb der Potentialbarriere ungleich null, sodass das Wellenpaket durchaus hinter der Barriere auffindbar sein kann. 

\noindent Bei der STM wird ein elektrisch leitender oder mit einem solchen Material überzogener Festkörper untersucht, indem eine Pt-Ir-Drahtspitze an eine Probe genähert wird, die dann über diese rastert. Es existiert also ein Valenzband in dem sich Elektronen der Fermienergie befinden. Die zu überwindende Potentialbarriere für den Tunneleffekt in diesem Versuch ist das Vakuum. Der Effekt ist nur für Abstände in atomischer Skala relevant, weswegen eine typische Probe der Größenordnung 100\,\r{A}\(^2\) ist und der Versuchsaufbau dementsprechend sensitiv gegenüber dem Abstand zwischen Spitze und Probe ist. Eine Genauigkeit von 1\,\r{A} für die Spitze kann beim Anlegen einer Spannung von \(0,1\)V erreicht werden.

\noindent Wird eine Spitze in die Nähe einer Probe mit diesen Eigenschaften gebracht, kann der Tunneleffekt für die Elektronen der Spitze oder für die der Probe eintreten, was davon abhänging ist, welches Vorzeichen die an die Spitze angelegte Spannung hat. Es entsteht ein Tunnelstrom, der gemessen werden kann.

\noindent Der Tunnelstrom fällt exponentiell mit dem Abstand \(d\) und der Wurzel der Austrittsarbeit \(\Phi\) des Festkörpers. Dies ist die Energie, die mindestens aufgewandt werden muss, um ein Elektron aus einem ungeladenen Festkörper zu lösen. 

\begin{equation}
I_\text{T}=\frac{U}{d}\exp{(-Kd\sqrt{\Phi})},\quad K\approx1,025\frac{1}{\text{\r{A}}\,\text{eV}^\frac12}
\end{equation}

\noindent Daraus ergeben sich direkt die zwei Hauptschwierigkeiten dieses Versuchs. Zum einen muss der gesamte Versuchsaufbau bezüglich mechanischer Vibrationen in der Größenordnung von \(<1\)\,\r{A} gedämpft werden, da der Abstand \(d\approx0,5\)\,\r{A} anfällig auf sehr schwache Schwingungen ist.  

\noindent Zum anderen muss die Drahtspitze möglichst spitz und sauber präpariert sein. Das heißt der Draht wird vor der benutzung mit 2-Propanol von lipopholem Dreck befreit.


\subsection{Fehlerquellen}
\end{document}

%\begin{table}[H]
%\begin{center}
%\begin{tabular}{c|c}
%\toprule
%\midrule
%\bottomrule
%\end{tabular}
%\caption{Tabellen Überschrift}
%\label{}
%\end{center}
%\end{table}