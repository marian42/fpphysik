\section{Einleitung}
Dieser Versuch behandelt die Funktion der Rastertunnelmikroskopie. Der Name leitet sich daraus ab, dass Festkörperproben mit einer Spitze abgerastert werden und daraus Bilder der Oberfläche der jeweiligen Festkörper aufgenommen werden können, welches durch den Tunneleffekt realisiert wird.

\noindent Im Folgenden wird der theoretische Hintergrund der STM (scanning tunneling microscopy) erläutert und die Messung für Proben HOPG (highly oriented pyrolytic graphite) und Gold ausgewertet.