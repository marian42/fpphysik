\section{Diskussion}

Die Messung des Gittervektors von Graphit mithilfe der Rastertunnelmikroskopie erfolgte relativ genau.
Der Literaturwert liegt im Fehlerbereich des gemessenen Wertes.
Der gemessene Wert weicht um $8,1\%$ vom Literaturwert ab.

Bei der Messung mit Gold enthalten die Aufnahmen starkes Rauschen.
Abgesehen vom Rauschen ergeben sich weitere Artefakte am Rand der Aufnahme, die größer sind als die eigentlichen Stufenkanten.
Diese könnten durch eine gekrümmte Probenoberfläche oder durch Fehler im Rastertunnelmikroskop erklärt werden.

Die Höhe der Stufenkante im aufgenommenen Profil stimmt mit dem Literaturwert überein.

Die Qualität der Messungen mit dem verwendeten Rastertunnelmikroskop hängt von einer Reihe von Faktoren ab, die unterschiedlich gut kontrolliert werden können.
Beispielsweise können Verunreinigungen des Aufbaus oder eine schlechte Messspitze das Ergebnis verschlechtern.
Ein weiterer Faktor, der verbessert werden könnte, ist die Einstellung der PID-Regler.