\section{Diskussion}

Die Messung des Gittervektors von Graphit mithilfe der Rastertunnelmikroskopie erfolgte sehr genau.
Der Literaturwert liegt im Fehlerbereich des gemessenen Wertes.
Beim Vergleich der Scans, die nach oben bzw. nach unten aufgenommen wurden, fällt auf, dass der gemessene Gitterwinkel sich stark unterscheidet, jedoch innerhalb der beiden Messungen in etwa gleich ist.
Dies lässt auf einen Messfehler schließen, der von der Y-Bewegungsrichtung abhängig ist, etwa ein Hystereseeffekt der Piezokristalle.

Bei der Messung mit Gold enthalten die Aufnahmen starkes Rauschen.
Es musste zunächst die Farbskala so angepasst werden, dass Stufenkanten erkennbar werden.
Abgesehen vom Rauschen ergeben sich weitere Artefakte am Rand der Aufnahme, die größer sind als die eigentlichen Stufenkanten.
Diese könnten durch eine gekrümmte Probenoberfläche oder durch Fehler im Rastertunnelmikroskop erklärt werden.

Die Qualität der Messungen mit dem verwendeten Rastertunnelmikroskop hängt von einer Reihe von Faktoren ab, die unterschiedlich gut kontrolliert werden können.
Beispielsweise können Verunreinigungen des Aufbaus oder eine schlechte Messspitze das Ergebnis verschlechtern.
Ein Vergleich mit vorherigen Scans des gleichen Geräts zeigt jedoch, dass durchaus genauere Messungen möglich sind.