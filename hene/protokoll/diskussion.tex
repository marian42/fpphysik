\section{Diskussion}
%STABILITÄT
Der maximale Resonatorabstand, konnte nicht erreicht werden, da die optische Schiene nicht lang genug ist. Die Länge des Laserrohrs verhindert, dass der minimale Abstand erreicht werden kann. Allerdings konnte das Minimum der Intensität an der Nullstelle gemessen werden. Der Wert \(L_{\text{NST}}=135,4\text{cm}\), der sich aus Wert zehn aus Tabelle \ref{tab:t1} weicht nur um \(3,3\%\) vom theoretischen Wert ab.

\noindent Für den Aufbau mit einem planaren Spiegel konnte der minimale bzw. maximale Abstand aus obigen Gründen ebenso nicht gemessen werden. Jedoch ist anhand der Messdaten deutlich ein linearer Abfall der Intensität zu erkennen, was auch der Theorie entspricht.

\noindent Die Stabilität eines Resonators hängt von der Form der Spiegel ab und ob es sich um konfokale Spiegel handelt. Dies gibt sich auch in den Messdaten zu erkennen, da in der Konstellation mit einem planaren Spiegel nicht mal die Hälfte der Intensität erzielt werden konnte, wie bei zwei gleich gekrümmten Spiegeln.
%TEM-MODEN
\noindent Der Draht, der bei der Vermessung der Moden verwendet wurde, dient als Störung der Schwingung, da schon bei geringen Störungen, die Mode gewechselt wird.
%POLARISATION
\noindent Im Bezug auf die Polarisation, sind Brewsterfenster von grundlegender Bedeutung. Sie stehen im Brewsterwinkel, so dass das zur Grenzfläche senkrecht polarisierte Licht zum größten Teil reflektiert wird, jedoch das dazu parallel polarisierte Licht nicht geschwächt wird.

\noindent Den Laser verlässt so im Idealfall linearpolarisiertes Licht. Der Einfluss der Resonatorspiegel auf die Polarisation ist im Vergleich zum Polarisationsfilter und den Brewsterfenstern sehr gering und kann somit vernachlässigt werden.
%WELLENLÄNGE
Die berechnete Wellenlänge weicht um \(1,4\%\) vom theoretischen Wert \(\lambda_{\text{th}}=632,8\text{nm}\) ab. Dies ist sehr gut.

\noindent Generelle Abweichungen der Messwerte von der Theorie sind durch nicht hinreichende Justage und Erschütterungen der optischen Achse zu erklären, da der Aufbau demgegenüber empfindlich ist.