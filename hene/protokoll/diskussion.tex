\section{Diskussion}
%STABILITÄT
Für \(r_1=r_2=1,4\)m konnte die maximale Resonatorlänge nicht erreicht werden, da die optische Schiene nicht lang genug ist. Die Länge des Laserrohrs verhindert, dass der minimale Abstand erreicht werden kann. Für den Aufbau mit einem planaren Spiegel konnte der minimale Abstand aus obigem Grund ebenso nicht gemessen werden. Jedoch ist anhand der Messdaten deutlich ein linearer Abfall der Intensität zu erkennen, was auch der Theorie entspricht. Die Stabilität eines Resonators hängt von der Form der Spiegel ab und ob es sich um einen konfokalen Resonator handelt. Dies gibt sich auch in den Messdaten zu erkennen, da in der Konstellation mit einem planaren Spiegel nicht mal die Hälfte der Intensität erzielt werden konnte, wie bei zwei gleich gekrümmten Spiegeln.

%TEM-MODEN
\noindent Die Intensitätsverteilung für die \(\text{TEM}_{10}\) Mode sollte eigentlich symmetrisch sein, jedoch ist dies nicht gelungen. Dafür sind mehrere Gründe zu nennen. Zum Einen kann die Photodiode leicht angewinkelt gewesen sein, so dass sich die Maxima in eine Richtung verlagern. Zum Anderen kann der Draht nicht bezüglich jeder Ausrichtung die gleiche Breite haben, also kein perfekter Zylinder sein. Deswegen konnte die Ausgleichsrechnung nur für zwei verschiedene Offset-Parameter \(d_0\) und \(d_1\) durchgeführt werden.

%POLARISATION
\noindent Im Bezug auf die Polarisation, sind Brewsterfenster von grundlegender Bedeutung. Sie stehen bezüglich der optischen Achse im Brewsterwinkel, so dass das zur Einfallsebene senkrecht polarisierte Licht zum größten Teil reflektiert wird, jedoch das dazu parallel polarisierte Licht nicht geschwächt wird. Die Verluste werden somit minimiert, weshalb die Brewsterfenster für den Laser wichtig sind. Den Laser verlässt so im Idealfall linear polarisiertes Licht.

%WELLENLÄNGE
\noindent Die berechnete Wellenlänge weicht um \(8\)nm vom theoretischen Wert \(\lambda_{\text{th}}=632,8\text{nm}\) ab. Die Abweichung ist groß, aber dies ist nicht dem Messverfahren geschuldet, sondern der Messung. Generelle Abweichungen der Messwerte von der Theorie sind durch nicht hinreichende Justage des Laserrohrs und der Spiegel und Erschütterungen der optischen Achse zu erklären, da der Aufbau demgegenüber empfindlich ist.