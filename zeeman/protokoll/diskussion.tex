\section{Diskussion}

Tabelle \ref{tab:theoriemessung} zeigt die gemessenen Landé-Faktoren verglichen mit den therietisch bestimmten Werten.
\begin{table}
	\centering
	\begin{tabular}{r r r}
		\toprule
		& $\Delta m g_\text{J}$, Theorie & $\Delta m g_\text{J}$, gemessen \\
		\midrule	
		rot, $\sigma$ & $1$ & \si{1,12 \pm 0,05} \\
		blau, $\sigma$ & $\frac{3}{2}$, $2$ & \si{1,99 \pm 0,24} \\
		blau, $\pi$ & $\frac{1}{2}$ & \si{0,57 \pm 0,05} \\
		\bottomrule
	\end{tabular}
	\caption{Theorie und Messung der Landé-Faktoren}
	\label{tab:theoriemessung}
\end{table}
Die Theoriewerte liegen in etwa im Fehlerbereich der gemessenen Werte.
Es fällt auf, dass im Fall der blauen $\sigma$-Linie der Fehler deutlich größer ist.
Dies ist  dadurch zu erklären, dass hier zwei Linien beobachtet werden, die die Lummer-Gehrcke-Platte nicht auflösen kann.

Mögliche Fehlerquellen beim Versuchsaufbau sind eine ungenaue Justierung der Optik, Fehler beim Ablesen der Linienabstände in den Bilddateien und Ungenauigkeiten bei der Aufnahme der Messreihe für die Magnetfeldstärke.