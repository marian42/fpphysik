\section{Diskussion}
Die berechneten Werte unterscheiden sich von der Literatur in einigen Werten. Es gibt einige ''Ausreißer'' in der isobaren Molwärme und somit auch in der isochoren wie zum Beispiel Messwert sieben und 15. Diese entstehen durch die relativ große Trägheit des Heizwiderstandes. Dadurch ist es sehr schwierig den Widerstand des Gehäuses auf den, der Probe zu bringen. Diese sollten nicht mehr als drei Ohm von einander verschieden sein.

\noindent Dies ist auch der Grund für das Verschwinden eines Wertes der Debye-Temperatur, die durch die Anpassung bestimmt wurde. Dadruch ergibt sich ein viel zu kleiner Wert für diese Temperatur. Die Abweichung zum Theoretischen Wert ist 45,6\%.

\noindent Ein weiterer Grund für die Messunsicherheiten ist die Wärmestrahlung, die nie komplett verhindert bzw. isoliert werden kann.