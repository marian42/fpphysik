\section{Diskussion}
Die berechneten Werte unterscheiden sich von der Literatur in einigen Werten. Es gibt einige ''Ausreißer'' in der isobaren Molwärme und somit auch in der isochoren wie zum Beispiel Messwert sieben und 15. Diese entstehen durch die relativ große Trägheit des Heizwiderstandes und die Wärmestrahlung, die nie komplett verhindert bzw. isoliert werden kann. Dadurch ist es sehr schwierig den Widerstand des Gehäuses auf dem der Probe zu halten. Diese sollten nicht mehr als drei Ohm von einander verschieden sein.

\noindent Dies ist auch der Grund für das Verschwinden des Wertes \(C_V=25,04\) der Debye-Temperatur. Dieser wurde allerdings bei der Mittelwertsberechnung der Debye-Temperatur nicht berücksichtigt. Es ergibt sich ein etwas größerer Wert für die Debye-Temperatur. Die Abweichung zum Theoretischen Wert ist mit 33,6\% sehr hoch. Auch die Abweichung des Mittelwerts der Debye-Temperatur ist mit 118,63 sehr hoch. Die Messunsicherheiten jener entstehen über eine lange Fortpflanzung von Fehlern, da für die Debye-Temperatur die isochore Molwärme benötigt wird, für die wiederum die isobare Molwärme und der Ausdehnungskoeffizient sowie die Temperatur benötigt wird usw.. 

\noindent Der Ausdehnungskoeffizient hat wegen der Wahl des Interpolationspolynoms einen großen Fehler. Vielleicht könnten durch eine andere Wahl der Interpolation bessere Ergebnisse erzielt werden.