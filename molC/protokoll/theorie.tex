\section{Theorie}

Die Wärmekapazität eines Körpers gibt an, wie groß die Temperaturänderung beim Zuführen von Wärmeenergie ist.
Die Wärmekapazität wird bei konstantem Volumen oder bei konstantem Druck angegeben.

\begin{align}
	C_v &= \left( \frac{\partial U}{\partial T} \right)_V \\
	C_p &= \left( \frac{\partial U}{\partial T} \right)_p
\end{align}

Teilt man die Wärmekapazität durch die molare Masse der Probe, erhält man die molare Wärmekapazität, auch Molwärme genannt, des Körpers.
Die Molwärme ist vom verwendeten Material und von der Temperatur abhängig.

Im Folgenden werden drei Modelle beschrieben, die die Molwärme von Festkörpern erklären.

\subsection{Das klassische Modell}
Die klassische Theorie geht davon aus, dass Atome im Festkörper auf drei Achsen harmonisch um eine feste Ruhelage schwingen.
Es wird außerdem das Äquipartitionstheorem zugrunde gelegt, das besagt, dass sich die Wärmeenergie gleichmäßig auf die Freiheitsgrade des Atoms verteilt.

Für die Molwärme ergibt sich aus diesen Annahmen:
\begin{align}
	C_v = 3R
\end{align}
Dabei ist $R$ die allgemeine Gaskonstante.

\subsection{Das Einstein-Modell}
Beim Einstein-Modell wird die Energiequantelung der Atomschwingungen berücksichtigt.
Es wird angenommen, dass alle Atome mit der gleichen Kreisfrequenz schwingen, die ein Vielfaches von $\bar{h}\omega$ ist.

Im Einstein-Modell ergibt sich für die Molwärme:
\begin{align}
	C_v = 3R \frac{\bar{h}^2 \omega^2}{k^2}\frac{1}{T^2}\frac{\exp \left(\frac{\bar{h}\omega}{k T}\right)}{\left(\exp\left(\frac{\bar{h}\omega}{k T}\right) - 1\right)^2}
\end{align}

Dabei ist $R$ die allgemeine Gaskonstante, $k$ die Boltzmann-Konstante und $\omega$ die Phononenkreisfrequenz.

\subsection{Das Debye-Modell}

Im Debye-Modell wird statt einer festen Schwingungsfrequenz ein Frequenzspektrum angenommen.
Für das Frequenzspektrum wird weiterhin angenommen, dass keine Dispersion stattfindet und dass die Frequenz nicht von der Ausbreitungsrichtung abhängt.

Im Debye-Modell ergibt sich für die Molwärme:

\begin{align}
	C_v = 9R \left(\frac{T}{\theta_D}\right)^3 \int_0^\frac{\theta_D}{T} \frac{x^4 e^x}{\left(e^x - 1\right)^2} \mathrm{d}x
\end{align}

Dabei ist R die allgemeine Gaskonstante, $x = \frac{\bar{h} \omega}{kT}$, und $\theta_D = \frac{\bar{h} \omega_D}{k}$ die Debye-Temperatur.
Die Debye-Temperatur ist die einzige Materialkonstante der Molwärme im Debye-Modell.