\section{Theorie}

\subsection{Wärmekapazität}
Die Wärmekapazität eines Körpers gibt an, wie groß die Temperaturänderung beim Zuführen von Wärmeenergie ist.
Sie wird bei konstantem Volumen oder bei konstantem Druck angegeben:

\begin{align}
	C_v &= \left( \frac{\partial U}{\partial T} \right)_V \\
	\label{cp}
	C_p &= \left( \frac{\partial U}{\partial T} \right)_p.
\end{align}

Die Wärmekapazität geteilt durch die molare Masse der Probe heißt molare Wärmekapazität oder Molwärme des Körpers.
Sie ist vom verwendeten Material und von der Temperatur abhängig.

Wird Wärme bei konstanter Leistung zugeführt, lässt sich die molare Wärmekapazität \eqref{cp} insbesondere schreiben als
\begin{equation}
\label{cp_mol}
	C_p=\frac{m_\text{mol}}{m}\frac{E}{\Delta T}\quad.
\end{equation}

Mit der Korrekturformel
\begin{equation}
	\label{kf}
	C_V=C_p-9\alpha^2\kappa V_0T
\end{equation}
kann zwischen den Molwärmen bei konstantem Volumen und Druck umgerechnet werden.
Dabei ist  $\alpha$ der Ausdehnungskoeffizient und $\kappa$ der Kompressionsmodul.

Im Folgenden werden drei theoretische Modelle beschrieben, die die Molwärme von Festkörpern erklären.

\subsubsection{Das klassische Modell}
Die klassische Theorie geht davon aus, dass Atome im Festkörper auf drei Achsen harmonisch um eine feste Ruhelage schwingen.
Es wird außerdem das Äquipartitionstheorem zugrunde gelegt, das besagt, dass sich die Wärmeenergie gleichmäßig auf die Freiheitsgrade des Atoms verteilt.

Im Festkörper hat jedes Atom pro Freiheitsgrad die mittlere Energie $\frac{1}{2}kT$.
Bei drei Freiheitsgraden und Berücksichtigung von kinetischer und potentieller Energie hat ein Atom das Sechsfache dieser Energie:
\begin{align}
	\langle u \rangle = 6\frac{1}{2}kT.
\end{align}
Für eine Teilchenmenge von einem Mol ist die Energie
\begin{align}
	U = 3kN_L \cdot T = 3RT.
\end{align}
Dabei ist $R$ die allgemeine Gaskonstante und $N_L$ die Loschmidtsche Zahl.

Für die Molwärme ergibt sich daraus:
\begin{align}
	C_V = \left(\frac{\partial U}{\partial T}\right)_V = 3R.
\end{align}

Dieser konstante Zusammenhang heißt ,,Dulong-Petit-Gesetzt" und macht nur für große Temperaturen eine korrekte Vorhersage.

\subsubsection{Das Einstein-Modell}
Beim Einstein-Modell wird gegenüber dem klassischen Modell die Energiequantelung der Atomschwingungen berücksichtigt.
Es wird angenommen, dass alle Atome mit der gleichen Kreisfrequenz schwingen, die ein Vielfaches von $\hbar\omega$ ist, und auch nur gequantelte Energien aufnehmen oder abgeben können.

Ausgehend von einer Boltzmann-Verteilung für die Wahrscheinlichkeiten des Vorkommens von Energiequantenzahlen ergibt sich für die mittlere Atomenergie
\begin{align}
	\langle u \rangle = \frac{\hbar \omega}{\exp \left( \frac{\hbar \omega}{kT} \right)-1}.
\end{align}

Daraus ergibt sich für die Molwärme:
\begin{align}
	C_V = \frac{\partial}{\partial T} 3N_L \langle u \rangle = 3R \frac{\hbar^2 \omega^2}{k^2}\frac{1}{T^2}\frac{\exp \left(\frac{\hbar\omega}{k T}\right)}{\left(\exp\left(\frac{\hbar\omega}{k T}\right) - 1\right)^2}
\end{align}

Dabei ist $N_L$ die Loschmidtsche Zahl und $k$ die Boltzmann-Konstante.

Wie im klassischen Modell wird für große Temperaturen eine konstante Molwärme $C_V = 3R$ vorhergesagt.
Für kleine Temperaturen nimmt die Wärmekapazität exponentiell ab, im Grenzfall $T \rightarrow 0$ ist $C_V = 0$.
Dazwischen ist die Vorhersage des Einstein-Modells für die Wärmekapazität kleiner als der tatsächliche Wert.

\subsubsection{Das Debye-Modell}

Im Debye-Modell wird statt einer festen Schwingungsfrequenz ein Frequenzspektrum angenommen.
Mit dem gleichen Ansatz wie im Einstein-Modell wird nun über alle Frequenzen integriert und die Atomenergie mit der Spektralverteilung $Z(\omega)$ gewichtet:
\begin{align}
	\label{debye1}
	C_V = \frac{\partial}{\partial T} \int\limits_{0}^{\omega_\mathrm{max}} Z(\omega) \frac{\hbar \omega}{\exp \left( \frac{\hbar \omega}{kT} \right)-1} \mathrm{d}\omega.
\end{align}
Für die Spektralverteilung wird ein Modell verwendet, dass einige Vereinfachungen einführt.
Es wird angenommen, dass im Kristall keine Dispersion stattfindet und dass die Phasengeschwindigkeit einer Welle unabhängig von der Ausbreitungsrichtung ist.

Aus diesen Voraussetzungen und weiteren geometrischen Betrachtungen kann die spektrale Dichte der Gitterschwingungen angegeben werden:
\begin{align}
	\label{spektraldichte1}
	Z(\omega) \mathrm{d}\omega = \frac{L^3}{2 \pi^2} \omega^2 \left(\frac{1}{v_\mathrm{l}^3} + \frac{2}{v_\mathrm{tr}^3}\right)\mathrm{d}\omega.
\end{align}
Die Probe wird als würfelförmig angenommen mit der Kantenlänge $L$.
$v_\mathrm{l}$ und $v_\mathrm{tr}$ sind die Phasengeschwindigkeiten der longitudinalen und transversalen Gitterschwingungen.
Für den Fall, dass Longitudinalwellen und Transversalwellen die gleiche Phasengeschwindigkeit haben, $v_\mathrm{l} = v_\mathrm{tr}$, vereinfacht sich \eqref{spektraldichte1} zu
\begin{align}
	\label{spektraldichte2}
	Z(\omega)\mathrm{d}\omega = \frac{3L^3}{2 \pi^2v^3}\omega^2 \mathrm{d}\omega.
\end{align}
In einem endlich großen Kristall mit $N_L$ Atomen kann es nur $3N_L$ Eigenschwingungen gegben.
Dadurch ist auch die Schwingfrequenz begrenzt.
Die größtmögliche Schwingfrequenz heißt Debye-Frequenz $\omega_D$.
Dies ist auch die maximale Frequenz und somit die Grenze für das Integral in \eqref{debye1}.
Aus der Anzahl der Eigenschwingungen lässt sich eine Gleichung für die Debye-Frequenz aufstellen.
\begin{align}
	\label{debyeintegral}
	\int_{0}^{\omega_D}Z(\omega)\mathrm{d}\omega = 3N_L
\end{align}
Mit der Spektraldichte \eqref{spektraldichte1} lässt sich daraus die Debye-Frequenz ausrechnen:
\begin{align}
	\label{debyefreq}
	\omega_D^3 = \frac{6 \pi^2 v^3 N_L}{L^3}
\end{align}
Mit der vereinfachten Spektraldichte \eqref{spektraldichte2} ergibt sich
\begin{align}
	\omega_D^3 = \frac{18 \pi^2 N_L}{L^3} \frac{1}{\left(\frac{1}{v_\mathrm{l}^3} + \frac{2}{v_\mathrm{tr}^3}\right)}.s
\end{align}
Die Gleichungen \eqref{spektraldichte1} und \eqref{spektraldichte2} können vereinfacht werden zu
\begin{align}
	Z(\omega)\mathrm{d}\omega = \frac{9N_L}{\omega_D^3} \omega^2 \mathrm{d}\omega.
\end{align}

So ergibt sich für die Molwärme aus \eqref{debye1}:
\begin{align}
	C_V & = \frac{\partial}{\partial T}\frac{9N_L \hbar}{\omega_D^3}  \int\limits_{0}^{\omega_\mathrm{max}} \frac{\omega^3}{\exp \left( \frac{\hbar \omega}{kT} \right)-1} \mathrm{d}\omega \\
	& = 9R \left(\frac{T}{\theta_D}\right)^3 \int_0^\frac{\theta_D}{T} \frac{x^4 e^x}{\left(e^x - 1\right)^2} \mathrm{d}x
\end{align}

Dabei ist R die allgemeine Gaskonstante, $x = \frac{\hbar \omega}{kT}$, und $\theta_D = \frac{\hbar \omega_D}{k}$ die Debye-Temperatur.
Die Debye-Temperatur ist die einzige Materialkonstante der Molwärme im Debye-Modell.

Für große Temperaturen verhält sich das Debye-Modell wie das Einstein-Modell.
Bei kleineren Temperaturen ist es jedoch genauer:
\begin{align}
	C_V \propto T^3.
\end{align}

\subsection{Elektrische Bauteile}
Im Versuch wir eine Heizwicklung verwendet.
Für die elektrische Leistung gilt allgemein
\begin{align}
	P = U \cdot I.
\end{align}
Bei konstanter Leistung ist die abgegebene Energie
\begin{align}
	\label{eenergie}
	E = P \cdot \Delta t = U \cdot I \cdot \Delta t.
\end{align}
Weiterhin werden zur Messung von Temperaturen Pt-100-Widerstände benutzt.
Diese haben einen temperaturabhängigen Widerstand.
Mit folgender Gleichung\cite{v47} kann die Temperatur berechnet werden:
\begin{equation}
	\label{pt100}
	T=0,00134\cdot R^2+2,296\cdot R-243,02
\end{equation}
Dabei ist $T$ in \SI{}{\celsius} und $R$ in \SI{}{\ohm}.